\section{Watcher Alarms}

The SDSS TPM sends a subset of its native alarms to the Watcher. 
Within IOP a tpmHandler process is periodically run that uses the 
EPICS Channel Access client interface (via the UNIX executable caGet1) 
to obtain the alarm severities of a number of TPM records. These records 
are then grouped by category, for example M1\_TEMP for all primary mirror 
thermal management problems.

The file containing the Watcher alarm definitions is
{\tt<tpm\_root>}/rtApp/Db/tpmAlarms.db. 

The alarm severities are, in order of increasing criticality, NO\_ALARM, 
MINOR, MAJOR, and INVALID. The INVALID state is set when communications 
with a device are lost, i.e. when we have no information about a system.

The IOP tpmHandler finds the maximum severity within each group, 
and populates the tpmError array with that severity and the tpmData 
array with the corresponding catagory. These two arrays are then 
transmitted to the Watcher.

The following section describe the individual TPM channels that IOP and 
the Watcher monitor, the Watcher alarm category they contribute to, 
and a short description.

\subsection{Alarm Group: M1\_TEMP}

The M1\_TEMP alarm group (Table \ref{tbl-m1temp})  checks 
for thermal gradients within the 
primary mirror of the SDSS 2.5 meter and excessive differences 
between the mirror and its environment. The quoted limits are in
degC and are those originally specified by Jim Gunn. The TPM 
is presently running with these limits uniformly increased by a factor of 3.

\begin{deluxetable}{lll}
\tablecaption{M1 Temperature Alarms\label{tbl-m1temp}}
\tablehead{
\colhead{Channel} &
\colhead{Description} &
\colhead{Limits} 
}
\startdata
    tpm\_TM\_M1Upper\_FB & Excessive front to back gradient in upper half of M1 & -0.2, 0.2 \\
    tpm\_TM\_M1Lower\_FB & Excessive front to back gradient in lower half of M1 & -0.2, 0.2 \\
    tpm\_TM\_M1Right\_FB & Excessive front to back gradient in right half of M1 & -0.2, 0.2 \\
    tpm\_TM\_M1Left\_FB & Excessive front to back gradient in left half of M1 & -0.2, 0.2 \\
    tpm\_TM\_M1\_FB & Excessive front to back gradient in M1 & -0.2, 0.2 \\
    tpm\_TM\_M1\_Rgrad & Excessive radial (center to edge) gradient in M1 & -1.0, 1.0 \\
    tpm\_TM\_M1\_Telair & Excessive mirror to ambient difference for M1 & -2.0, 2.0 \\
    tpm\_TM\_M1\_Denv & Excessive mirror to PMSS difference for M1 & -2.0, 2.0 \\
\enddata
\end{deluxetable}
    

\subsection{Alarm Group: M2\_TEMP}

The M2\_TEMP alarm group (Table \ref{tbl-m2temp}) 
checks for thermal gradients within 
the secondary mirror of the SDSS 2.5 meter and excessive 
differences between the mirror and its environment. The quoted
limits are in degC and are those originally specified by Jim 
Gunn. The TPM is presently running with these limits uniformly 
increased by a factor of 3.

\begin{deluxetable}{lll}
\tablecaption{M2 Temperature Alarms\label{tbl-m2temp}}
\tablehead{
\colhead{Channel} &
\colhead{Description} &
\colhead{Limits} 
}
\startdata
    tpm\_TM\_M2Upper\_FB & Excessive front to back gradient in upper half of M2 &  -0.2, 0.2 \\
    tpm\_TM\_M2Lower\_FB & Excessive front to back gradient in lower half of M2 & -0.2, 0.2 \\
    tpm\_TM\_M2Right\_FB & Excessive front to back gradient in right half of M2 & -0.2, 0.2 \\
    tpm\_TM\_M2Left\_FB & Excessive front to back gradient in left half of M2 & -0.2, 0.2 \\
    tpm\_TM\_M2\_FB & Excessive front to back gradient in M2 & -0.2, 0.2 \\
    tpm\_TM\_M2\_Rgrad & Excessive radial (center to edge) gradient in M2 & -1.0, 1.0 \\
\enddata
\end{deluxetable}

\subsection{Alarm Group: ENV\_TEMP}

The ENV\_TEMP alarm group (Table \ref{tbl-envtemp}) checks for 
excessive differences between 
the telescope structure, the rotating floor, the lower enclosure, the 
return air from cooling system, and the environments. The quoted 
limits are in degC and are those originally specified by Jim Gunn. 
The TPM is presently running with these limits uniformly increased 
by a factor of 3.
  
\begin{deluxetable}{lll}
\tablecaption{Environmental Temperature Alarms\label{tbl-envtemp}}
\tablehead{
\colhead{Channel} &
\colhead{Description} &
\colhead{Limits} 
}
\startdata
    tpm\_TM\_Floor\_Telair & Excessive difference between floor and ambient & -2.0, 2.0 \\
    tpm\_TM\_Return\_Telair & Excessive difference between cooling return air and ambient & -2.0, 2.0 \\
    tpm\_TM\_Lowair\_Telair & Excessive difference between lower enclosure and ambient & -3.0, 3.0 \\
\enddata
\end{deluxetable}  

\subsection{Alarm Group: TRACKING}

The TRACKING alarm group (Table \ref{tbl-tracking}) checks the
magnitudes of the axis 
servo errors when the telescope is assumed to be tracking. The 
alarm limits are in encoder counts and define the MINOR (warning level) 
and MAJOR (problems) conditions.

\begin{deluxetable}{lll}
\tablecaption{Tracking Alarms\label{tbl-tracking}}
\tablehead{
\colhead{Channel} &
\colhead{Description} &
\colhead{Limits} 
}
\startdata
    tpm\_alttrackchk & Altitude servo error exceeds limits & -10, -7, 7, 10 \\ 
    & & when v $<$ 3200 counts/sec \\
    tpm\_aztrackchk & Azimuth servo error exceeds limits & -10, -7, 7, 10 \\ 
    & & when v $<$ 3200 counts/sec \\
    tpm\_rottrackchk &  Rotator servo error exceeds limits & -2800, -1400, 1400, 2800  \\ 
    & & when v $<$ 3200 counts/sec \\
\enddata
\end{deluxetable}

\subsection{Alarm Group: CURRENTS}

The CURRENTS alarm group (Table \ref{tbl-currents}) 
checks the reported motor currents 
for possible problems. The present limits (in counts) are 
arbitrary. These alarms have never been enabled.

\begin{deluxetable}{lll}
\tablecaption{Current Limit Alarms\label{tbl-currents}}
\tablehead{
\colhead{Channel} &
\colhead{Description} &
\colhead{Limits} 
}
\startdata
    tpm\_ALMTRC1 & Altitude drive motor \#1 current exceeds limits & 1000, 2000 \\
    tpm\_ALMTRC2 & Altitude drive motor \#2 current exceeds limits & 1000, 2000 \\
    tpm\_AZMTRC1 & Azimuth drive motor \#1 current exceeds limits & 1000, 2000 \\
    tpm\_AZMTRC2 & Azimuth drive motor \#2 current exceeds limits & 1000, 2000 \\
    tpm\_ROMTRC & Rotator drive motor current exceeds limits & 1000, 2000 \\
\enddata
\end{deluxetable}

\subsection{Alarm Group: DEWARS}
 
The DEWARS alarm group (Table \ref{tbl-dewar}) 
checks for the imager and spectrograph 
dewar weights dropping below safe levels. These alarms also check 
the status of the serial communications between the TPM and the 
dewar scales. The alarm limits define the MINOR (warning level) 
and MAJOR (problem) weight limits in pounds.

\begin{deluxetable}{lll}
\tablecaption{Dewar Alarms\label{tbl-dewar}}
\tablehead{
\colhead{Channel} &
\colhead{Description} &
\colhead{Limits} 
}
\startdata
    tpm\_Ndewar\_spectro & Spectro dewar weight low or communications failure & 50, 75 and comm \\
    tpm\_Sdewar\_imager & Imager dewar weight low or communications failure & 50, 75 and comm \\
\enddata
\end{deluxetable}

\subsection {Alarm Group: MIGS}
The MIGS alarm group (Table \ref{tbl-mig}) 
monitors the state of the 
serial communications between the TPM and the M1/M2 MIGS.

\begin{deluxetable}{lll}
\tablecaption{MIG Alarms\label{tbl-mig}}
\tablehead{
\colhead{Channel} &
\colhead{Description} &
\colhead{Limits} 
}
\startdata
    tpm\_MIG1A & M1 MIG communications failure & comm \\
    tpm\_MIG2A & M2 MIG communications failure & comm \\
\enddata
\end{deluxetable}

\subsection{Alarm Group: GALILS}
The GALILS alarm group (Table \ref{tbl-galil})
monitors the state of the 
serial communications between the TPM and the M1/M2 Galils. The 
mirror misposition alarms proposed by Jim Gunn have not been implemented.

\begin{deluxetable}{lll}
\tablecaption{Galil Alarms\label{tbl-galil}}
\tablehead{
\colhead{Channel} &
\colhead{Description} &
\colhead{Limits} 
}
\startdata
    tpm\_GAL1 & M1 Galil communications failure & comm \\
    tpm\_GAL2 & M2 Galil communications failure & comm \\
\enddata
\end{deluxetable}

\subsection{Alarm Group: TMICRO}
The TMICRO alarm group (Table \ref{tbl-tmicro}) 
monitors the state of the 
serial communication between the TPM and the Temperature Micro in 
addition to checking for broken or faulty sensors. A sensor fault condition is assumed if the readback is less than -50 degC.
\begin{deluxetable}{lll}
\tablecaption{TMICRO Alarms\label{tbl-tmicro}}
\tablehead{
\colhead{Channel} &
\colhead{Description} &
\colhead{Limits} 
}
\startdata
    tpm\_TM\_M0B0 & TMICRO communications failures & comm \\
    tpm\_TM\_M0B0Chans & Faulty temperature sensors & -50 \\
    tpm\_TM\_M0B1Chans & Faulty temperature sensors & -50 \\
    tpm\_TM\_M0B2Chans & Faulty temperature sensors & -50 \\
    tpm\_TM\_M0B3Chans & Faulty temperature sensors & -50 \\
    tpm\_TM\_M0B4Chans & Faulty temperature sensors & -50 \\
    tpm\_TM\_M0B5Chans & Faulty temperature sensors & -50 \\
    tpm\_TM\_M0B6Chans & Faulty temperature sensors & -50 \\
    tpm\_TM\_M0B7Chans & Faulty temperature sensors & -50 \\
    tpm\_TM\_M1B0Chans & Faulty temperature sensors & -50 \\
    tpm\_TM\_M1B1Chans & Faulty temperature sensors & -50 \\
    tpm\_TM\_M1B2Chans & Faulty temperature sensors & -50 \\ 
    tpm\_TM\_M1B3Chans & Faulty temperature sensors & -50 \\
    tpm\_TM\_M1B4Chans & Faulty temperature sensors & -50 \\
    tpm\_TM\_M1B5Chans & Faulty temperature sensors & -50 \\ 
    tpm\_TM\_M1B6Chans & Faulty temperature sensors & -50 \\
    tpm\_TM\_M1B7Chans & Faulty temperature sensors & -50 \\
    tpm\_TM\_M2B0Chans & Faulty temperature sensors & -50 \\
    tpm\_TM\_M2B1Chans & Faulty temperature sensors & -50 \\
    tpm\_TM\_M2B2Chans & Faulty temperature sensors & -50 \\
    tpm\_TM\_M2B3Chans & Faulty temperature sensors & -50 \\
    tpm\_TM\_M2B4Chans & Faulty temperature sensors & -50 \\
    tpm\_TM\_M2B5Chans & Faulty temperature sensors & -50 \\ 
    tpm\_TM\_M2B6Chans & Faulty temperature sensors & -50 \\
    tpm\_TM\_M2B7Chans & Faulty temperature sensors & -50 \\
\enddata
\end{deluxetable}

