\section{Information for Developers}

Within this section we describe the structure of the TPM
code distribution and the build procedure.
The TPM is available from FermiLab CVS as the module
{\bf tpm}. 

\subsection{Directory Structure}

Within the installed TPM distribution are found directories
that are either checked out from CVS or are generated during
the build process. Table \ref{tbl-directory} provides a summary
of these directories with their origin and contents listed.

\begin{deluxetable}{lll}
\tablecaption{TPM Directory Structure\label{tbl-directory}}
\tablehead{
\colhead{Directory} &
\colhead{Origin} &
\colhead{Description} 
}
\startdata
ChanArch & CVS & Channel Archiver configuration and documents \\
bin & Built & Workstation and IOC executables \\
commonApp & CVS & Device drivers and custom EPICS software \\
config & CVS & EPICS application development system configuration \\
db & Built & EPICS databases \\
dbd & Built & EPICS database descriptions \\
doc & CVS & Documentation \\
include & Built & C language include files \\
iocBoot & CVS & vxWorks startup scripts \\
javalib & Source & Compiled Java code - not used \\
logApp & CVS & Pre-EPICS TPM software {\bf obsolete} \\
mcp & CVS & Contains {\tt data\_collection.h} \\
opi & Built & EPICS EDD/DM real-time display files \\
rtApp & CVS & TPM application specific software \\
rtApp/Db & CVS & EPICS database definitions \\
rtApp/opi & CVS & EPICS EDD/DM real-time display files \\
rtApp/src & CVS & C source, sequencer scripts, master {\tt .dbd} files \\
slaApp & CVS & SLALIB source \\
ups & CVS & FermiLab UPS configuration files \\
\enddata
\end{deluxetable}

\subsection{Building the TPM}

Creation of a new version of the TPM is performed using the
Makefile found in the top level directory. The default action
is to compile the source and to create the databases. Other
supported actions are to clean a previous build and to install
a new build. The TPM uses the EPICS application configuration
and build system and thus relies on {\tt gnumake} rather than 
{\tt sdssmake}.

After a update of the TPM has been created and committed to
CVS it needs to be labeled with a CVS rtag.
TPM version names have the form
{\tt tpm\_vXX\_YY\_ZZ} where {\tt XX} is the major release
number, {\tt YY} is the minor release number, and {\tt ZZ}
indicates position along a branch. To date all TPM releases
have been along the main line of development with no
branches required, therefore the nominal value of {\tt ZZ} is ``00''.

It is project software configuration policy that new release
numbers be generated when {\bf any} changes are made to an
operational system.

The build phase for the TPM is:\\
\$ setup tpm \\
\$ setup sdsscvs \\

\$ ssh-agent bash \\
\$ ssh-add \\

\$ cvs rtag {\tt <version>} tpm \\

\$ cvs export -r {\tt <version>} -d {\tt <mydir>} tpm \\
\$ cd {\tt <mydir>} \\
\$ gnumake \\

The install option moves the build products into a target
directory specified by the \$TPM\_DIR environment
variable. 

The installation phase is:\\
\$ export TPM\_DIR=/p/tpm/{\tt<new\_version>} \\

\$ gnumake install \\
\$ ups declare {-c,-t} -f IRIX+6 -m tpm.table \\
-r /p/tpm/{\tt<new\_version>} tpm {\tt<new\_version>} \\
\# use -c for current \\
\# use -t for test \\
\# use null if normal \\

\subsection{Rebooting the TPM}
 
Changing TPM versions requires first executing the
{\tt switchTPM} script. This script creates a series of soft links between
top level subdirectories under \$TPM\_DIR under the TPM's
standard location of {\tt /p/tpmbase/}. This root
directory is that hard-coded in the TPM's executable, 
display, archiver configuration, documentation, and
system startup paths. The soft links are generated for the
following directories: db, dbd, bin, opi, iocBoot, and ChanArch.

\$ setup tpm {\tt<new\_version>} \\
\$ switchTPM 

Rebooting the TPM IOCs requires access to their vxWorks consoles.
This is done using the terminal server connections to the two
IOCs. 

To connect to SDSSTPM: {\tt telnet t-g-sdss-2 4000} \\
To reboot: [tpm:]\$ {\tt reboot} \\

To connect to SDSSTPM2: {\tt telent t-g-sdss-2 3300} \\
To reboot: [tpm2:]\$ {\tt reboot} \\

\subsection{Location of WWW-based Documentation}

The apache webserver running on sdsshost provides access to
specific versions of the TPM documentation based on the
declared ``current'' and ``test'' versions with ups.
The {\tt /p/www\_server/apache-1.3.26/conf/access.conf.hurl}
file maps the URL {\tt tpm} to the directory
{\tt <<upsroot tpm>>/ChanArch/tpmwww} and the URL
{\tt tpmTest} to {\tt <<upsroot tpm -t>>/ChanArch/tpmwww}.


\subsection{Archiver cron Process}

The TPM executes a cron process on sdsshost that checks the
status of the Channel Archiver and updates the WWW-based
temperature sensor displays. The source files for the
cron jon are found in {\tt <tpm\_root>/ChanArch/bin}.

The primary function of status
check and display update is executed every quarter hour.
If the ArchiveEngine fails to respond to a query on port 4812
within 5 seconds then the script attempts to restart the 
ChannelArchiver process. Temperature sensor data is collected
from the TPM via channel access and new Web pages are generated.
These pages are copied into both the UPS declared ``current''
and ``test'' releases. {\bf Jon Brinkmann indicates this
explicit dependence on UPS has caused problems with the 
cron jobs}.

Once a day, at 11:10 hours local time, the cron process stops
the ChannelArchiver to roll over to a new MJD directory. The
value of the MJD is acquired in real-time from the TPM
channel {\tt tpm\_MJD}.
