%%
%% Beginning of file 'sample.tex'
%%
%% Modified 03 Nov 99
%%
%% This is a sample manuscript marked up using the
%% AASTeX v5.0 LaTeX 2e macros.

%% The first piece of markup in an AASTeX v5.0 document
%% is the \documentclass command. LaTeX will ignore
%% any data that comes before this command.

%% The command below calls the default manuscript style,
%% which will produce a double-spaced document on one column.
%% Examples of commands for other substyles follow. Use
%% whichever is most appropriate for your purposes.

\documentclass[preprint]{aastex}
\usepackage{amsmath}
\usepackage{html}

%% preprint produces a one-column, single-spaced document:

% \documentclass[preprint]{aastex}

%% preprint2 produces a double-column, single-spaced document:

% \documentclass[preprint2]{aastex}

%% If you want to create your own macros, you can do so
%% using \newcommand. Your macros should appear before
%% the \begin{document} command.
%%
%% If you are submitting to a journal that translates manuscripts
%% into SGML, you need to follow certain guidelines when preparing
%% your macros. See the AASTeX v5.0 Author Guide
%% for information.



%% If you wish, you may supply running head information, although
%% this information may be modified by the editorial offices.
%% The left head contains a list of authors,
%% usually a maximum of three (otherwise use et al.).  The right
%% head is a modified title of up to roughly 44 characters.  Running heads
%% will not print in the manuscript style.


%% This is the end of the preamble.  Indicate the beginning of the
%% paper itself with \begin{document}.

\begin{document}

%% LaTeX will automatically break titles if they run longer than
%% one line. However, you may use \\ to force a line break if
%% you desire.

\title{The SDSS Telescope Performance Monitor}

\author{Peregrine M. McGehee\\5 March 2004}

The document is maintained by Peregrine M. McGehee at Los Alamos
National Laboratory and the Department of Astronomy at New Mexico
State University. Questions and comments should be directed to
{\it peregrin\_at\_apo.nmsu.edu}.

A gzip-ed Postscript version of this document is available at
\htmladdnormallink{tpm.ps.gz}{../tpm.ps.gz}. 
The \LaTeX{} source, figures, and build scripts are in SDSS CVS
as part of the {\tt tpm} module. 

%% Use \author, \affil, and the \and command to format
%% author and affiliation information.
%% Note that \email has replaced the old \authoremail command
%% from AASTeX v4.0. You can use \email to mark an email address
%% anywhere in the paper, not just in the front matter.
%% As in the title, you can use \\ to force line breaks.

\section{Introduction}
The Telescope Performance Monitor displays and archives engineering data
using the infrastructure supplied by the Experimental Physics and 
Industrial Control System (EPICS) toolkit.
Detailed information concerning EPICS can be found at the project home
page at Argonne National Laboratory:
\htmladdnormallink{http://epics.aps.anl.gov/epics}{http://epics.aps.anl.gov/epics}

The real-time component of the TPM is implemented on two Motorola 
MV162 (Petra) VMEbus computers running the vxWorks operating system 
(v5.1.1) and version R3.13.5 of EPICS iocCore. These two MV162s, 
hereafter referred to as Input-Output Controllers (IOCs), acquire 
information about the Sloan telescope and ancillary systems using a variety of interfaces. 

This document begins with a discussion of the information needed
for developers followed by an overview of the operator displays.
The breakdown of the TPM into logical subsystems is presented. Next the
physical interfaces used by the TPM are described.

Within the section on alarms we categorize those native TPM alarms
that are transmitted to IOP and watcher. The document concludes
with a series of tables organized by logical subsystem that
describe all of the TPM process variables.

\section{Information for Developers}

Within this section we describe the structure of the TPM
code distribution and the build procedure.
The TPM is available from FermiLab CVS as the module
{\bf tpm}. 

\subsection{Directory Structure}

Within the installed TPM distribution are found directories
that are either checked out from CVS or are generated during
the build process. Table \ref{tbl-directory} provides a summary
of these directories with their origin and contents listed.

\begin{deluxetable}{lll}
\tablecaption{TPM Directory Structure\label{tbl-directory}}
\tablehead{
\colhead{Directory} &
\colhead{Origin} &
\colhead{Description} 
}
\startdata
ChanArch & CVS & Channel Archiver configuration and documents \\
bin & Built & Workstation and IOC executables \\
commonApp & CVS & Device drivers and custom EPICS software \\
config & CVS & EPICS application development system configuration \\
db & Built & EPICS databases \\
dbd & Built & EPICS database descriptions \\
doc & CVS & Documentation \\
include & Built & C language include files \\
iocBoot & CVS & vxWorks startup scripts \\
javalib & Source & Compiled Java code - not used \\
logApp & CVS & Pre-EPICS TPM software {\bf obsolete} \\
mcp & CVS & Contains {\tt data\_collection.h} \\
opi & Built & EPICS EDD/DM real-time display files \\
rtApp & CVS & TPM application specific software \\
rtApp/Db & CVS & EPICS database definitions \\
rtApp/opi & CVS & EPICS EDD/DM real-time display files \\
rtApp/src & CVS & C source, sequencer scripts, master {\tt .dbd} files \\
slaApp & CVS & SLALIB source \\
ups & CVS & FermiLab UPS configuration files \\
\enddata
\end{deluxetable}

\subsection{Building the TPM}

Creation of a new version of the TPM is performed using the
Makefile found in the top level directory. The default action
is to compile the source and to create the databases. Other
supported actions are to clean a previous build and to install
a new build. The TPM uses the EPICS application configuration
and build system and thus relies on {\tt gnumake} rather than 
{\tt sdssmake}.

After a update of the TPM has been created and committed to
CVS it needs to be labeled with a CVS rtag.
TPM version names have the form
{\tt tpm\_vXX\_YY\_ZZ} where {\tt XX} is the major release
number, {\tt YY} is the minor release number, and {\tt ZZ}
indicates position along a branch. To date all TPM releases
have been along the main line of development with no
branches required, therefore the nominal value of {\tt ZZ} is ``00''.

It is project software configuration policy that new release
numbers be generated when {\bf any} changes are made to an
operational system.

The build phase for the TPM is:\\
\$ setup tpm \\
\$ setup sdsscvs \\

\$ ssh-agent bash \\
\$ ssh-add \\

\$ cvs rtag {\tt <version>} tpm \\

\$ cvs export -r {\tt <version>} -d {\tt <mydir>} tpm \\
\$ cd {\tt <mydir>} \\
\$ gnumake \\

The install option moves the build products into a target
directory specified by the \$TPM\_DIR environment
variable. 

The installation phase is:\\
\$ export TPM\_DIR=/p/tpm/{\tt<new\_version>} \\

\$ gnumake install \\
\$ ups declare {-c,-t} -f IRIX+6 -m tpm.table \\
-r /p/tpm/{\tt<new\_version>} tpm {\tt<new\_version>} \\
\# use -c for current \\
\# use -t for test \\
\# use null if normal \\

\subsection{Rebooting the TPM}
 
Changing TPM versions requires first executing the
{\tt switchTPM} script. This script creates a series of soft links between
top level subdirectories under \$TPM\_DIR under the TPM's
standard location of {\tt /p/tpmbase/}. This root
directory is that hard-coded in the TPM's executable, 
display, archiver configuration, documentation, and
system startup paths. The soft links are generated for the
following directories: db, dbd, bin, opi, iocBoot, and ChanArch.

\$ setup tpm {\tt<new\_version>} \\
\$ switchTPM 

Rebooting the TPM IOCs requires access to their vxWorks consoles.
This is done using the terminal server connections to the two
IOCs. 

To connect to SDSSTPM: {\tt telnet t-g-sdss-2 4000} \\
To reboot: [tpm:]\$ {\tt reboot} \\

To connect to SDSSTPM2: {\tt telent t-g-sdss-2 3300} \\
To reboot: [tpm2:]\$ {\tt reboot} \\

\subsection{Location of WWW-based Documentation}

The apache webserver running on sdsshost provides access to
specific versions of the TPM documentation based on the
declared ``current'' and ``test'' versions with ups.
The {\tt /p/www\_server/apache-1.3.26/conf/access.conf.hurl}
file maps the URL {\tt tpm} to the directory
{\tt <<upsroot tpm>>/ChanArch/tpmwww} and the URL
{\tt tpmTest} to {\tt <<upsroot tpm -t>>/ChanArch/tpmwww}.


\subsection{Archiver cron Process}

The TPM executes a cron process on sdsshost that checks the
status of the Channel Archiver and updates the WWW-based
temperature sensor displays. The source files for the
cron jon are found in {\tt <tpm\_root>/ChanArch/bin}.

The primary function of status
check and display update is executed every quarter hour.
If the ArchiveEngine fails to respond to a query on port 4812
within 5 seconds then the script attempts to restart the 
ChannelArchiver process. Temperature sensor data is collected
from the TPM via channel access and new Web pages are generated.
These pages are copied into both the UPS declared ``current''
and ``test'' releases. {\bf Jon Brinkmann indicates this
explicit dependence on UPS has caused problems with the 
cron jobs}.

Once a day, at 11:10 hours local time, the cron process stops
the ChannelArchiver to roll over to a new MJD directory. The
value of the MJD is acquired in real-time from the TPM
channel {\tt tpm\_MJD}.


\section{Displays}

Within this section we describe the EDD/DM display
hierarchy. The top-level TPM real-time display is invoked 
using {\tt \$ dm tpm.dl} on sdsshost. The structure
of the display system is summarized in Tables
\ref{tbl-display1},
\ref{tbl-display2},
\ref{tbl-display3},
\ref{tbl-display4},
\ref{tbl-display5}, and \ref{tbl-display6}.

\begin{deluxetable}{ll}
\tablecaption{TPM Top-Level Display Hierarchy\label{tbl-display1}}
\tablehead{
\colhead{Name} &
\colhead{Description}
}
\startdata
About & Product information \\
Details & Menu to access subsystem displays \\
Temperature & Menu to access thermal management displays \\
System Checks & Menu to access test configurations \\
Interlocks & Menu to access display of PLC files \\
Time Series & Menu to invoke Stripcharts \\
Weather & Weather data
\enddata
\end{deluxetable}

\begin{deluxetable}{ll}
\tablecaption{Details - Subsystem Displays\label{tbl-display2}}
\tablehead{
\colhead{Name} &
\colhead{Description}
}
\startdata
Axis & Altitude, azimuth, and rotator drive information \\
PVTs & MCP and TCC Position, Velocity, and Time triplets \\
M1/M2 Galils & Galil motor controller information for mirrors \\
Amp/Slip & Drive amplifier and slip detection status \\
PMSS & Primary mirror support system readings \\
PLC Analog & Display of analog values read from PLCs
\enddata
\end{deluxetable}

\begin{deluxetable}{ll}
\tablecaption{Temperature Displays\label{tbl-display3}}
\tablehead{
\colhead{Name} &
\colhead{Description}
}
\startdata
Sensor Data & Temperature readings grouped by module \\
Graphics & Color-coded schematics of mirror and truss temperatures \\
Alarms & State of thermal management alarms \\
Enclosure Dew Pt & Status and alarm display for enclosure DP depression
\enddata
\end{deluxetable}

\begin{deluxetable}{ll}
\tablecaption{System Checks\label{tbl-display4}}
\tablehead{
\colhead{Name} &
\colhead{Description}
}
\startdata
Tests & Configuration information for system tests
\enddata
\end{deluxetable}

\begin{deluxetable}{ll}
\tablecaption{Interlocks\label{tbl-display5}}
\tablehead{
\colhead{Name} &
\colhead{Description}
}
\startdata
I1 & Allen-Bradley PLC file display \\
01 & Allen-Bradley PLC file display \\
I2 & Allen-Bradley PLC file display \\
02 & Allen-Bradley PLC file display \\
I3 & Allen-Bradley PLC file display \\
I4 & Allen-Bradley PLC file display \\
I5 & Allen-Bradley PLC file display \\
I6 & Allen-Bradley PLC file display \\
I7 & Allen-Bradley PLC file display \\
I8 & Allen-Bradley PLC file display \\
I9 & Allen-Bradley PLC file display \\
I10 & Allen-Bradley PLC file display \\
O11 & Allen-Bradley PLC file display \\
O12 & Allen-Bradley PLC file display \\
B3 & Allen-Bradley PLC file display \\
B10 & Allen-Bradley PLC file display 
\enddata
\end{deluxetable}

\begin{deluxetable}{ll}
\tablecaption{Time Series\label{tbl-display6}}
\tablehead{
\colhead{Name} &
\colhead{Description}
}
\startdata
Generic & StripTool with no predefined configuration \\
Position Errors & StripTool showing servo errors \\
Rotator & StripTool showing rotator axis data \\
Altitude & STripTool showing altitude axis data \\
Azimuth & StripTool showing azimuth axis data \\
Current & StripTool showing drive current readings
\enddata
\end{deluxetable}

\subsection{Screen Shots}

\begin{figure}
\caption{{\bf Need to display subregions as the resolution is poor 
for this version.}}
\plotone{tpm_dl.ps}
\end{figure}


\section{Subsystems}

The TPM is logically composed of a number of subsystems. Within
Table \ref{tbl-systems} we provide short descriptions of each of these
Each channel within the TPM is associated with a specific subsystem. 
The description field (DESC) within the each EPICS database record
contains the name of the associated subsystem using the syntax
``<subsystem>:<description>''. 

This use of the DESC field to 
hold information utilized by the documentation system often
results in the total string length exceeding the
29 character limit for this field. As a consequence 
error messages of the form ``Cannot set field DESC..'' 
appear during the
execution of {\tt dbLoadRecords} at IOC startup.
The error message is somewhat misleading because the string 
is truncated to 29 characters and then written into the real-time
database. This has no effect on IOC operation.

\begin{deluxetable}{ll} 
\tablecaption{TPM Logical Subsystems\label{tbl-systems}}
\tablehead{
\colhead{Subsystem} &
\colhead{Description}
}
\startdata
    ADMIN & Administrative information for sdsstpm and sdsstpm2 \\
    AMP & Drive amplifier data \\
    DEWAR &  Imager and Spectrograph 180l dewar weight and pressure \\
    DIO & Raw digital data from IP-470 DIO Industry Pack Modules \\
    GALIL & M1 and M2 mirror position information from GALIL motion controller \\
    MIG & M1 and M2 Mirror position information from Mitutoyo gauges \\
    MOTOR & Motor current and voltage data \\
    PLC & Allen-Bradley PLC-based interlock system data structures \\
    PMSS & Primary mirror support system data \\
    SERVO & Servo systems data: position, velocity, errors \\
    SHMEM & MCP-TPM shared memory interface configuration \\
    SLIP & Slip detection system \\
    THERMAL & Thermal monitoring system \\
    WEATHER & Dew point and humidity sensor data, wind speed and direction \\
\enddata
\end{deluxetable}



\section{Interfaces}

The SDSS Telescope Peformance Monitor acquires engineering data 
from a heterogeneous collection of interfaces. These interfaces 
and the data acquired from them are described in this document.

The real-time component of the Telescope Performance Monitor is 
implemented on two Motorola MV162 CPUs that run the vxWorks 
operating system and the EPICS control system tool-kit.

The machine sdsstpm shares a VMEbus crate with the Mount 
Control Processor. The physical interfaces utilized by sdsstpm 
to acquire data are summarized in Table \ref{tbl-tpmif}.
All serial communications are handled by sdsstpm2, which resides 
in a separate VMEbus crate. These interfaces are listed in
Table \ref{tbl-tpm2if}

The engineering drawings that document the TPM electronic
interfaces are available via the FermiLab DCS. 
{\bf Need list of drawing numbers and working access method}. 
The exception is the
documentation for the thermal monitoring
system which is found at
\htmladdnormallink{ftp://ftp.astro.princeton.edu/jeg/Thermometer}{ftp://ftp.astro.princeton.edu/jeg/Thermometer}.

\subsection{SDSSTPM Interfaces}
.
\begin{deluxetable}{ll}
\tablecaption{sdsstpm Physical Interfaces\label{tbl-tpmif}}
\tablehead{
\colhead{Interface} &
\colhead{Subsystems} 
}
\startdata
MCP-TPM Shared memory & MCP \\
& MEI Data \\
& Allen-Bradley PLC Data \\
Acromag IP470, 12-bit DIO: IP Slot A & Word 0: AMP:AZ1(0:7) \\
& Word 1: AMP:AZ1(9:11)/AZ2(0:3) \\
& Word 2: AMP:AZ2(4:11) \\
& Word 3: AMP:AL1(0:7) \\
& Word 4: AMP:AL1(9:11)/AL2(0:3) \\
& Word 5: AMP:AL2(4:11) \\
Acromag IP470, 12-bit DIO: IP Slot B & Word 0: AMP:ROT(0:7) \\
& Word 1: AMP:ROT(9:11) \\
& Word 2: SLIP(0:7) \\
& Word 3: SLIP(8:11) \\
& Word 4: SLIP MUX(0:3)[PMSS 4:7] \\
& Word 5: MCP SYNC(0) \\
Greensprings IP-16ADC: IP Slot C & Channel 0: PMSS \\
& Channel 1: Wind Speed \\
& Channel 2: Wind Direction \\
& Channel 3: Imager LN2 Pressure \\
& Channel 4: Spectrograph LN2 Pressure \\
\enddata
\end{deluxetable}

\subsection{SDSSTPM\#2 Interfaces}

\begin{deluxetable}{ll}
\tablecaption{sdsstpm2 Physical Interfaces\label{tbl-tpm2if}}
\tablehead{
\colhead{Interface} &
\colhead{Subsystems} 
}
\startdata
Serial (GreenSprings IP Octal Serial RS-232 Module): IP Slot D & Port 1: Temperature Micro \\
& Port 2: M1 Galil \\
& Port 3: M2 Galil \\
& Port 4: Spectrograph dewar scale \\
& Port 5: Imager dewar scale \\
& Port 6: M1 MIG \\
& Port 7: M2 MIG \\
& Port 8: *EMPTY* \\
Telnet to monts.apo.nmsu.edu terminal server & Port 2500: Humidity sensor \\
\enddata
\end{deluxetable}


# TPM-watcher alarms not otherwise archived.
# Temperature inhomogeneity tests
tpm_TM_M1Upper_FB	10
tpm_TM_M1Lower_FB	10
tpm_TM_M1Right_FB	10
tpm_TM_M1Left_FB	10
tpm_TM_M1_FB	10
tpm_TM_M1_Rgrad	10
tpm_TM_M1_Telair	10
tpm_TM_M1_Denv	10
tpm_TM_M2Upper_FB	10
tpm_TM_M2Lower_FB	10
tpm_TM_M2Right_FB	10
tpm_TM_M2Left_FB	10
tpm_TM_M2_FB	10
tpm_TM_M2_Rgrad	10
tpm_TM_Floor_Telair	10
tpm_TM_Return_Telair	10
tpm_TM_Lowair_Telair	10
# Track checking
# tests on servo errors, if vel > 3200 cnts/sec,
# test on -10,-7,7,10 error counts.
tpm_alttrackchk	0.05
tpm_aztrackchk	0.05
tpm_rottrackchk	0.05
# Drive currents
# test on -2000,-1000,1000,2000 counts
# already archived
#tpm_ALMTRC1
#tpm_ALMTRC2
#tpm_AZMTRC1
#tpm_AZMTRC2
#tpm_ROMTRC
# serial communications checks
tpm_Ndewar_spectro.SEVR	10
tpm_Sdewar_imager.SEVR	10
tpm_MIG1.SEVR	10
tpm_MIG2.SEVR	10
tpm_GAL1.SEVR	10
tpm_GAL2.SEVR	10
tpm_TM_M0B0.SEVR	10
# missing channel (T < 50 degC) tests
tpm_TM_M0B0Chans.SEVR	10
tpm_TM_M0B1Chans.SEVR	10
tpm_TM_M0B2Chans.SEVR	10
tpm_TM_M0B3Chans.SEVR	10
tpm_TM_M0B4Chans.SEVR	10
tpm_TM_M0B5Chans.SEVR	10
tpm_TM_M0B6Chans.SEVR	10
tpm_TM_M0B7Chans.SEVR	10
tpm_TM_M1B0Chans.SEVR	10
tpm_TM_M1B1Chans.SEVR	10
tpm_TM_M1B2Chans.SEVR	10
tpm_TM_M1B3Chans.SEVR	10
tpm_TM_M1B4Chans.SEVR	10
tpm_TM_M1B5Chans.SEVR	10
tpm_TM_M1B6Chans.SEVR	10
tpm_TM_M1B7Chans.SEVR	10
tpm_TM_M2B0Chans.SEVR	10
tpm_TM_M2B1Chans.SEVR	10
tpm_TM_M2B2Chans.SEVR	10
tpm_TM_M2B3Chans.SEVR	10
tpm_TM_M2B4Chans.SEVR	10
tpm_TM_M2B5Chans.SEVR	10
tpm_TM_M2B6Chans.SEVR	10
tpm_TM_M2B7Chans.SEVR	10
# Imager UPS status
tpm_UPSParse.VALA	10
tpm_UPSParse.VALB	10
tpm_UPSParse.VALC	10
tpm_UPSParse.VALD	10
tpm_UPSParse.VALE	10
tpm_UPSParse.VALF	10
tpm_UPSParse.VALG	10
tpm_UPSParse.VALH	10
tpm_UPSParse.VALI	10
tpm_UPSParse.VALJ	10
tpm_UPSParse.VALK	10
tpm_UPSParse.VALL	10
tpm_UPSParse.VALM	10
tpm_UPSParse.VALN	10
tpm_UPSParse.VALO	10


\section{Channel Reports By Subsystem}

To provide an overview of the individual TPM channels we present
a series of tables for each logical subsystem. The channels within
each subsystem are separated into those that are written into the
TPM logs by the EPICS channel archiver and those that are not.

For each channel we provide the full process variable name 
including the database record field. The default field name
is ``.VAL''. In addition to the process variable name we
give the scanning period within the EPICS database, the device
driver specification, and the channel description.

This section is automatically generated from the ASCII database
files found in {\tt<tpmbase>}/db and the ChannelArchiver
configuration files located in {\tt<tpmbase>}/ChanArch/config. The
documentation generation process uses a combination of bash scripts
and Tcl program driven by {\tt<tpmbase>}/doc/gentpmdoc.sh.

The calling sequence is:\\
\$ gentpmdoc.sh {\tt<tpm\_root>} \\
where {\tt<tpm\_root>} is the top of a TPM distribution directory, e.g.
/p/tpmbase or /usrdevel/peregrin/tpm.

 
\subsection{ADMIN}
 
This subsection is derived from the EPICS database
entries and the ChannelArchiver configuration files
via gentpmdoc.sh.
 
\subsection{AMP}
 
This subsection is derived from the EPICS database
entries and the ChannelArchiver configuration files
via gentpmdoc.sh.
 
\subsection{DEWAR}
 
This subsection is derived from the EPICS database
entries and the ChannelArchiver configuration files
via gentpmdoc.sh.
 
\subsection{DIO}
 
This subsection is derived from the EPICS database
entries and the ChannelArchiver configuration files
via gentpmdoc.sh.
 
\subsection{GALIL}
 
This subsection is derived from the EPICS database
entries and the ChannelArchiver configuration files
via gentpmdoc.sh.
 
\subsection{MIG}
 
This subsection is derived from the EPICS database
entries and the ChannelArchiver configuration files
via gentpmdoc.sh.
 
\subsection{MOTOR}
 
This subsection is derived from the EPICS database
entries and the ChannelArchiver configuration files
via gentpmdoc.sh.
 
\subsection{PLC}
 
This subsection is derived from the EPICS database
entries and the ChannelArchiver configuration files
via gentpmdoc.sh.
 
\subsection{PMSS}
 
This subsection is derived from the EPICS database
entries and the ChannelArchiver configuration files
via gentpmdoc.sh.
 
\subsection{SERVO}
 
This subsection is derived from the EPICS database
entries and the ChannelArchiver configuration files
via gentpmdoc.sh.
 
\subsection{SHMEM}
 
This subsection is derived from the EPICS database
entries and the ChannelArchiver configuration files
via gentpmdoc.sh.
 
\subsection{SLIP}
 
This subsection is derived from the EPICS database
entries and the ChannelArchiver configuration files
via gentpmdoc.sh.
 
\subsection{THERMAL}
 
This subsection is derived from the EPICS database
entries and the ChannelArchiver configuration files
via gentpmdoc.sh.
 
\subsection{WEATHER}
 
This subsection is derived from the EPICS database
entries and the ChannelArchiver configuration files
via gentpmdoc.sh.


\end{document}
