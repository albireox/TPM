\section{Subsystems}

The TPM is logically composed of a number of subsystems. Within
Table \ref{tbl-systems} we provide short descriptions of each of these
Each channel within the TPM is associated with a specific subsystem. 
The description field (DESC) within the each EPICS database record
contains the name of the associated subsystem using the syntax
``<subsystem>:<description>''. 

This use of the DESC field to 
hold information utilized by the documentation system often
results in the total string length exceeding the
29 character limit for this field. As a consequence 
error messages of the form ``Cannot set field DESC..'' 
appear during the
execution of {\tt dbLoadRecords} at IOC startup.
The error message is somewhat misleading because the string 
is truncated to 29 characters and then written into the real-time
database. This has no effect on IOC operation.

\begin{deluxetable}{ll} 
\tablecaption{TPM Logical Subsystems\label{tbl-systems}}
\tablehead{
\colhead{Subsystem} &
\colhead{Description}
}
\startdata
    ADMIN & Administrative information for sdsstpm and sdsstpm2 \\
    AMP & Drive amplifier data \\
    DEWAR &  Imager and Spectrograph 180l dewar weight and pressure \\
    DIO & Raw digital data from IP-470 DIO Industry Pack Modules \\
    GALIL & M1 and M2 mirror position information from GALIL motion controller \\
    MIG & M1 and M2 Mirror position information from Mitutoyo gauges \\
    MOTOR & Motor current and voltage data \\
    PLC & Allen-Bradley PLC-based interlock system data structures \\
    PMSS & Primary mirror support system data \\
    SERVO & Servo systems data: position, velocity, errors \\
    SHMEM & MCP-TPM shared memory interface configuration \\
    SLIP & Slip detection system \\
    THERMAL & Thermal monitoring system \\
    WEATHER & Dew point and humidity sensor data, wind speed and direction \\
\enddata
\end{deluxetable}

